\section{Game analysis}
This section is devoted to an in-depth quality-in-use assessment of
Jikiban, where the basic theory and concepts will be lifted from
several articles taking a scientific approach to game design theory.

\subsection{Playability}
In \cite{sanchez09} the concept ``playability'' is defined as:

\begin{quotation}
The degree to which a specific player achieve specific game goals with
effectiveness, efficiency, flexibility, security and, especially,
satisfaction in a playable context of use.
\end{quotation}

The layout will first delineate the different concepts used and then
proceed to an analysis of how these concepts helped shape and define
Jikiban as an interactive entertainment system.


\subsubsection{Playability theory}
Beyond a broad definition of playability an important dichotomy in
perspective is put forth by R. Hunicke
et. Al. \cite{hunicke01-04}. Namely that the designer has the
diametrically opposite view of the game compared to that of the
player, this dichotomy is illustrated in figure \ref{fig:MDA}.

\begin{itemize}
\item \emph{Mechanics} (M) describes the particular components of the game, at
  the level of data representation and algorithms.
\item \emph{Dynamics} (D) describes the run-time behavior of the mechanics
  acting on player inputs and each others’ outputs over time.
\item \emph{Aesthetics} (A) describes the desirable emotional responses evoked
  in the player, when she interacts with the game system.
\end{itemize}

\figurepng{MDA}{Figure 1: The perspective dichotomy of games.  Source
is Hunicke et. Al. \cite{hunicke01-04}}


Thus the impact on the aesthetics and the dynamics of any implemented
rule, mechanic, feature or piece of content must be imagined from the
perspective of a player or assessed though playtesting.

This dichotomy underlines the necessity for player feedback and as
such any game development process should be iterative with playtesting
as a key step, section \ref{player-feedback} will explain our
approach used for playtesting. In addition, this dichotomy obfuscates
the term playability, which now has two distinct meanings.

In \cite{nacke09} these two meanings are referred to as Playability
and Player Experience. In that Playability is the interaction between
the designer and the game, whereas Player Experience is the players
interaction with said game, see figure
\ref{fig:player-xp-vs-playability}.\\

\figurepng{player-xp-vs-playability}{The interfaces between player,
  game and game designer show that playability is directed toward
  evaluating game design, whereas player experience has to be analyzed
  in the player-game interaction process. Source is Nacke
  el. Al. \cite{nacke09}  }


The next sections will explore the different concepts put forth in
figure \ref{fig:MDA} in the context of Jikiban. Starting from the design
perspective different methods for evaluating Playability will be
investigated. Then moving onto a player driven view, which will shed
light on the Player Experience. Lastly the way feedback was used to
shape the game through playtesting is explained.

\subsection{Heuristics for Evaluating Playability (HEP)}
Heuristic refers to experience-based, trial and error techniques for
problem solving, learning, and discovery.

Based on research from the game research community \cite{desurvire08,
  hunicke01-04, sanchez09}, a set of heuristics were gathered, these
were developed and refined specifically for games. The heuristics fell
into four general areas:

\begin{itemize}

\item Gameplay
\begin{itemize}
\item Is the set of problems and challenges a player must face to win a game.
\end{itemize}

\item Usability
\begin{itemize}
\item Addresses the interface and encompasses the elements the player
  utilizes to interact with the game.q
\end{itemize}

\item Mechanics
\begin{itemize}
\item Involves the programming that provides the structure by which
  units interact with the environment.
\end{itemize}

\item Game story
\begin{itemize}
\item Includes all plot and character development.
\end{itemize}
\end{itemize}

The following sections will explore these topics in greater detail and
delineate their application in Jikiban.

\subsubsection{Gameplay}
The main challenge of Jikiban relies mainly on the player learning to
think in a new way. The player essentially has two distinct tasks, one
based on mental deduction, the other being an execution challenge.
These two task are not as trivial as they sound however. Since a new
feature or concept is introduced in every level the mental task is
either how to interact with the new game object or how to reconfigure
your idea of what you can do to include a solution to the present
problem. While the execution challenge, due the turn-based nature of
the game, is execution in the sense of making everything line up
perfectly, which will require both memory and imagination. In addition
to these two main driving forces, the complexity naturally increases
as the player has to incorporate ever more intricate concepts.

This concept of approachable depth will discourage casual gamers from
instantly giving up and hardcore gamers from scoffing at the ease of
progression. The idea of depth rather than difficulty is also
ingrained in the fact that there may be several ways to solve a level,
thereby given the player room to solve the problem in her own
way. While constantly providing her with tools to reduce impossible
situations to manageable challenges. The hope is that this natural
flow to the player’s ever more intricate model of the game will engage
the player, thus educating to better entertain.

It should be noted that this all comes at the expense of having a
bottleneck in level design, which is what necessitated the design of a
level editor for a prototype of the game.


\subsubsection{Usability}
Games as a medium is interactive by nature and as such requires
actions by the player. Many game designers thus tend to frontload
their games with tutorials in order to avoid overwhelming the player.
Wherein they introduce almost every aspect of the game and often do
this through text. Counter intuitively this often leads to the player
often entirely skipping the tutorial for the very reason it was there.

When designing Jikiban this issue naturally had to be addressed and
since it is a puzzlegame at its core, one obvious example to follow is
that of Portal, created by Valve Corporation [2007]. In Jikiban most
of the game is technically tutorialesque, in that the game is subdived
into a host of segments wherein new features or concepts are
introduced sequentially. This means that the game features a rich
puzzle with lots of interlocking systems which the player will
understand without the need for a manual.

When it comes to the interface Jikiban is lagging to a somewhat
significant degree. The game is a prototype and thus common features
such as scrolling text to convey controls are still missing. This
resulted in most players having to look up the controls in the help
menu, which is an undesired, but common misgiving of video-games. This
problem could be solved in a whole host of ways, but is acceptable
since the medium of a final game is undetermined. Had the game for
instance been implemented on a smartphone the available control would
be clearly visible on the touchscreen.

\subsubsection{Mechanics}
The function of every mechanic and the core system itself is
specifically chosen to be as transparent as possible. The fact that
any configuration that seems even slightly wrong will cause a time
paradox, this is facilitated through the use of a grid and distinct
turns, wherein everything acts simultaneously. In addition to this
every game object is designed to have a very clear function. The
theory behind this methodology is to ensure that players feel that the
game is fair.

\subsubsection{Game story}
The first point of order when discussing game story is that the very
term is misleading. In this I quote game developer James Portnow:
%FIXME, missing cite portnow?

\begin{quotation}
Video game story telling is not just about the writing or the
dialog. Because video games are an interactive medium the game play
and the mechanics of the world are just as important to telling the
story. Narrative, not writing is what we should be talking about.
\end{quotation}

Thus this topic will be referred to as narrative from here on in.

In this topic the approach of the old 8-bit era has been taken. The
narrative is presented through the graphics, the sound and the actions
taken. The fact that the game is entirely text-less widens the player
base to including young children. As Jikiban is a puzzlegame at heart
though the narrative is not an essential part of the experience.

A few things are reinforced though; the light graphics of a little
girl wandering around a hedge maze with campfires blocking the way
looking for gold coins instills a light tone to the game that anyone
can enjoy.  Where most games tend to kill off the main player
character as an emphasis on failure Jikiban aims for a less gruesome
appeal. In Jikiban death is replaced by time paradoxes, any ‘death’ of
the player character simply could not have happened. In addition to
this the turn-based system lends the game a different kind of danger
in that everything can be deduced, but thinking many turns ahead may
prove difficult.

The narrative of Jikiban is presented in the form of novelty, as the
game progresses new concept and features are constantly added, while
the player character seemingly has no goal other than finding the next
coin.

\subsection{Player experience}
The senior producer at the game studio Dice suggested\cite{hagen10}:

\begin{quotation}
You close your eyes and think: What do I find fun? It’s as simple as that.
\end{quotation}

This is a common approach in the industry and is a fitting reminder
that the industry is still young, Hagen called this method
autobiographical design \cite{hagen10}. Others however, have paved the
way for an empirical scientific approach. Hunicke et. Al.  proposes a
framework in which to describe the Player experience of any
game\cite{hunicke01-04}. These include:

\begin{itemize}
\item \emph{Sensation} - Game as sense-pleasure
\item \emph{Fantasy} - Game as make-believe
\item \emph{Narrative} - Game as drama
\item \emph{Challenge} - Game as obstacle course

\item \emph{Fellowship} % of the ring
 - Game as social framework % ^W^Wthe quest to destroy the one ring to rule them all.
\item \emph{Discovery} - Game as uncharted territory
\item \emph{Expression} - Game as self-discovery
\item \emph{Submission} - Game as pastime
\end{itemize}

Most games will only include a subset of these to a significant degree
and Jikiban is no exception. Jikiban as a single-player puzzle-game
relies almost exclusively on Challenge.

%TODO: review
The core engagement of Jikiban is the drive to understand and learn
how to think in a new way. Jikiban features a relentless learning
curve that never lets up. Every new level features new concepts that
the player will have to integrate into what she already knows in order
to progress. This easy to learn, but hard to master learning curve, in
a turn based system, is designed to provide a fair environment. In which
the player never feels overwhelmed, while promoting creative thought
followed by the engineering work to facilitate the idea. This
procrastination of reward should make every completed level feel like
a victory, rather than the inevitable outcome of a work session.

In order for this approachable depth concept to work, very clear cut
mechanics were needed. These very transparent mechanics combined with
the patient deterministic nature of the game places the blame and the
pride fully with the player.

These design concepts do have the unfortunately consequence of forcing
the players to try elaborate ideas which may fail. To ameliorate this
effect the music was chosen to be happy and patient, while trying not
to be monotone. This is in opposition with many puzzle games which
increase the pace of the music in order to amplify the urgency.


\section{Player feedback}
\label{player-feedback}
As mentioned previously feedback in this project is gathered solely
through observations during playtesting and subsequent dialog with the
test subjects.

Since playtesting starts early in this prototyping process, early
playtesting was performed using blackboards, paper and the designers
themselves. This part of the process will not be discussed here.  This
section will focus on playtesting performed with subjects unassociated
with the project.  It should be noted that playtesting is not
equivalent to bug testing, all bug testing was done by developers.

% TODO[NT], check the playtesting methods have been described first.
Bugs randomly encountered during playtests are naturally shifted to
bug testing phases.  The playtesting method used during this project
is the method proposed by Sánchez el. Al. \cite{sanchez09} combined
with the method proposed by Portnow et. Al. \cite{portnow}.

Since all test subjects have been adult gamer friends of the
developers, different preventative measures have been applied in order
to minimize inevitable biases. These encompass:

\begin{itemize}
\item Not explaining the game beforehand.
\begin{itemize}
\item This is essential since the game should be self-explanatory.
\end{itemize}

\item Not providing any help other than essential, yet-to-be-implemented features.
\begin{itemize}
\item For example what controls are available.
\end{itemize}

\item Talk to the subject as sparsely as possible.
\item Do not trap the player with social pressure.
\item Do not reject any feedback and do not try to explain. Simply
  listen and note everything.
\begin{itemize}
\item Observe how the subject uses the game, what they naturally want
  to do.
\item Note where they get stuck and what needs better explanation.
\end{itemize}


\item Subjects generally propose solutions rather than pointing out
  problems. These need to be interpreted.
\item Listen carefully for any vocalizations, like a sigh or a
  gasp. If these are noted something was either very right or very
  wrong.
\item Try to note what the subject is looking at, focusing on or
  thinking about. Obviously this data will mostly have to be gathered
  through questions.
\item Use questionnaires and software hooks to gather metrics, like
  goal efficiency, about the game. A host of these metrics are
  proposed here \cite{sanchez09}.
\begin{itemize}
\item These methods were beyond the scope of this project.
\end{itemize}

\end{itemize}

\subsubsection{Playtests}
\label{sec:playtests}

Three iterations of playtests were carried out, with interleaved
design changes to the game. The first of these iterations was
initiated as soon as a working game was up and running. This section
will not go through all individual iterations, but will briefly
outline the most noteworthy findings.  Early playtests showed that
the game was largely unplayable. It should be noted that this early
version had significantly fewer features than the final version. Most
noteworthy

\begin{itemize}
\item Although the levels were solvable, they had never been playtested.
\item There were no hints at all, so the players had no help at all.
\end{itemize}

The players had trouble understanding the concepts of the game such as
how the wait function works in the context of multiple player
copies. Only the active character waits.  However, if the concept of the
past-selves is not well understood, this is not trivial. It becomes
odd that the other past-selves move regardless of what the active
character does. This problem was addressed through a change in the
slope of the learning curve in early levels.

The time machine was the hardest concept to understand. Since the
character was standing on it when entering, it was not apparent that
time had been reset. Now there were simply two versions of the player,
one of which was very hard to control (actually uncontrollable). Both
of these concepts were introduced in level 2 along with buttons and
gates, so the player basically gets overwhelmed early in the game and
gives up.

The player often had problems identifying the active player. There
were nothing to mark which player was actually being moved and the
other past-selves were hidden when standing on the same spot. The
result was that the player's timing was ruined forcing him to restart,
due to a mistake.

The enormous amount of counting required to get through a level often
deterred the players from completing it at all. Often they would find
the trick to the level and then skip it, since the rest was just dull
work. A player described it as having to program the level after
finding the solution. This issue was addressed using patient music,
sound effects on level completion and a score system which encourages
optimization of solutions.

Changes were made to give the player more help in the form of hints,
the hint was displayed below each level. This resulted in a complaint
that the hint was always visible and gave away the solution. On the
bright side the game could now be played to the end, without help, by
experienced gamer. The player would not have played it this much if he
hadn't been asked to though. The game lacked fun and a motivation
factor.

%TODO: align with Evolution part
To accommodate these problems, the hint was hidden behind a button.
Some of the levels were rewritten to incorporate one time buttons and make
a more smooth level progression using these. This enabled the gates
and one-time buttons to be introduced before the ability to go back in
time.
